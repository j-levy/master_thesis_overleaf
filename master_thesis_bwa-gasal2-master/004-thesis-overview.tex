This thesis is organised as follows.
In Chapter \ref{chap:background}, an overview of the background to understand DNA alignment is given. We also provide explanations about GPU computing and its usage in this application. This aims to show the reason why we chose this approach to solve the problem of DNA alignment.

Chapter \ref{chap:accel} contains details about the accelerator discussed in this thesis. While it originally supported few features, more functionality and flexibility are added and its integration to an existing aligning software is presented.

The implementation of said improvements is shown in Chapter \ref{chap:implementation}. In particular, we provide a detailed view of how we adapted the original software to include our accelerator. We also review the modifications we brought to our solution to integrate it successfully.

Chapter \ref{chap:measurements} presents the experimental setup used to test the implemented accelerator and also discusses the measurement results showing its performance and accuracy.

Finally we will conclude in Chapter \ref{chap:ccl} with the final thoughts and possible future works.