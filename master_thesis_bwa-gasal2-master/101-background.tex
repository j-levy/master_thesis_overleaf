DNA alignment comes at a time where sequencing machines become more and more available and affordable. The principle behind alignment is to find a match between \emph{reads} coming out of a sequencing machine in a given \emph{reference} genome. 

To achieve this in a timely manner, most software rely on the \emph{seed-and-extend} technique.

In the first step, the \emph{seeding} part consists in finding a substring of a given length that matches exactly in the read and in the reference. Depending on the parameter used in various software, a maximum of zero, one, or several mismatches can be allowed. For this step, it is necessary to look into the whole genome for a match, and an index is used. One indexing algorithm relies on the Burrows-Wheeler transform~\cite{BurrowsWheeler:align}, and is used in many DNA aligners today. During this step, multiple seeds can be found in various parts of the reference genome, and depending on how they overlap, they can be grouped in chains. Then for all chains, an extension is performed.

The extension part consists in trying to pursue the alignment started from the chain on both sides, left and right. If the chain is located at the beginning or the end of the sequence, only one alignment is needed. More refinements are possible, for example, detecting if the score is dropping when continuing the alignment, and cutting the calculation to avoid computing useless parts; or only computing probable aligning positions, and skipping the highly unlikely ones. %These algorithms will be detailed in the next section.

