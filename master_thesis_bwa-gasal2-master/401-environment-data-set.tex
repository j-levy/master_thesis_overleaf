

First we review the conditions in which we run our measurements.

\subsection{Environment definition}

The specifications of the machine used for the measurements is shown in Table~\ref{tbl:hwspecs}. We report hardware references in Table~\ref{tbl:hwspecs} and software specifications on Table~\ref{tbl:swspecs}.

\begin{table}[h!]
	\centering
	\begin{tabular}{|c|c|}
		\hline 
		\textbf{Hardware} & \textbf{Reference} \\ 
		\hline 
		CPU & 2 * Intel(R) Xeon(R) CPU E5-2620 v3 @ 2.40GHz~\cite{misc:xeon} \\ 
		\hline 
		RAM & 32 GB DDR3 \\ 
		\hline 
		GPU & NVIDIA GK110BGL - Tesla K40c~\cite{misc:k40c} \\ 
		\hline 
		VRAM & 12 GB GDDR5\\
		\hline
	\end{tabular} 
	\caption{Hardware specifications}
	\label{tbl:hwspecs}
\end{table}

	\bigskip
	
\begin{table}[h!]
	\centering
	\begin{tabular}{|C{0.2\textwidth}|C{0.7\textwidth}|}
		\hline 
		\textbf{Software} & \textbf{Reference} \\ 
		\hline 
		Operating System & Red Had Enterprise Linux 7 \\ 
		\hline 
		Linux kernel &  3.10.0-957.5.1.el7.x86\_64 \\ 
		\hline 
		C compiler & g++, GCC version 4.8.5 20150623 \\ 
		\hline 
		CUDA version & 10.1.105 \\ 
		\hline 
		C compiler options & \verb|-g -Wall -Wno-unused-function -O2 -msse4.2 -std=c++11 -fpermissive| \\ 
		\hline 
		NVCC compiler options & \verb|-c -g -O3 -std=c++11 -Xcompiler -Wall,-DMAX_SEQ_LEN=\$(MAX_SEQ\_LEN),-DN_CODE=\$(N_CODE) -Xptxas -Werror -lineinfo --ptxas-options=-v --default-stream per-thread| \\ 
		\hline 
		
		\end{tabular} 
\caption{Software specifications}
\label{tbl:swspecs}
\end{table}

We apply an optimisation level of -O2 for the GCC compiler since we noted that it gave significantly faster execution than -O0 and -O1. O3 optimisation level makes no difference in speed with respect to -O2. For NVCC (the NVIDIA CUDA compiler) we keep the -O3 optimisation level that was originally used in all GASAL2 papers~\cite{Ahmed:gasal}.

The row "NVCC compiler options" has two particular defines passed with -D. \verb|MAX_SEQ_LEN| corresponds the maximum size of the sequence for the extension kernel. As we have seen previously in Figure~\ref{fig:visualisation-aid-tile}, we have to store the score of a whole row (in yellow in the figure). This size is defined at compilation time to avoid dynamic memory allocation inside the kernel, which makes it noticeably faster. Having to define this number at compilation is not very problematic because we know that Illumina machines produce reads with fixed lengths. \verb|N_CODE| defines the value used for the unknown base "N", since each program can use its own. For example, the test program provided with GASAL2 registers the bases with their ASCII values, so the N code is \verb|0x4E|; but BWA-MEM codes A, C, T, G, and N with integer values from 0 to 4, the "N" code being 4.

All tests are run with 1 and 2 streams, and with 1, 2, 4, 6, 8, 10, and 12 threads. We will observe how overlapped execution provides faster results while using 2 streams. In addition, we run the program with only one stream to force it to wait for its completion before re-filling the data structure containing the query and target sequences to align: this allows to measure the actual kernel time, since the CPU is actively waiting for the kernel to complete. 

\subsection{Data sets}

For the tests, we use two paired-end data sets that are mapped to a reference genome. Each data set is constituted of two files. The first file contains the first read of the pair, which is the direct strand of DNA. The second file contains the second read in the pair, made by reading the strand from the other end, hence being the reverse-complement of the first strand.

Data set \#1 called "SRR150"~\cite{ncbi:srr150} is made of 150-base long paired-end DNA reads. Each file has 5.2 million sequences. These are the typical lengths that a state-of-the-art Illumina sequencer can produce. Data set \#2, that we called "SRR250"~\cite{ncbi:srr250} has DNA reads with 250 bases each. This pair of files contains 8.3 million sequences in each file.

The reference we used is the Genome Reference Consortium Human Build 37 (GRCh37), also named "hg19"~\cite{ncbi:hg19}. It is the genome of a male human being (with X and Y chromosomes) sequenced in 2009. This whole genome takes around 3.2GB in its plain text version. Before running the mapper, an index of reference genome is created with BWA-MEM indexing tool. The index is the Burrows-Wheeler transform of the reference genome and has size of about 4.8 gigabytes. The seed-extension stage of BWA-MEM does not require the index. It only requires the original reference genome FASTA file which is 2-bit packed, and hence around 800 megabytes large. Since the target sequences extracted from the reference genome on the CPU, the file is not copied to the GPU to accelerate the seed extension.

