Mapping is the process of finding the location in the reference genome where a DNA read probably belongs to. Since 99\% of the genome is the same in a specie, finding the location in the reference genome is similar to identify the position of a read in the sample DNA. Moreover the mapping must also provide the information about how well the read is aligned to the reference genome giving the exact location of SNPs and indels.  Mapping DNA reads is a complex task due to large sequencing data and reference genome size. Different types of mutations also increase the complexity of the mapping. In the case of the human, the reference genome size is around 3.4GB when stored in plain text, meaning, having the full sequence of letters A, T, C and G to represent the DNA. 


At the moment, DNA mapping represents the first genomics analysis steps of many DNA analysis approach in practice. In addition, mapping algorithms are rather time consuming, both due to the high complexity of the analysis involved as well as the amount of data that needs to be processed. In many cases, alignment can take between $30\%$ and $50\%$ of the total DNA analysis time.  Figure~\ref{fig:pipelineprocesstime} shows that the time taken by mapping, is about a third of the total pipeline time. Moreover, the time taken for this part is counted in thousands of CPU-core hours for this example data set. Therefore DNA mapping is a computational challenge and various  techniques to achieve it in a timely manner. The problem at stake is then to find an effective way to compute DNA alignment as fast as possible.

\begin{figure}[h]
	\centering
	\includegraphics[width=1\linewidth]{pipelineprocesstime}
	\caption{DNA pipeline process time share for a typical 30$\times$ coverage cancer DNA data set. The data set consists of three tumor samples and one normal tissue sample (time given in CPU-core hours). (from~\cite{HOUTGAST201854})}
	\label{fig:pipelineprocesstime}
\end{figure}


% refer to that paper : https://www.sciencedirect.com/science/article/pii/S1476927118301555 and also show the figure in the paper (copy it and use a reference)
